%CS-113 S18 HW-6
%Released: 30-March-2018
%Deadline: 13-April-2018 7.00 pm
%Authors: Abdullah Zafar, Waqar Saleem.


\documentclass[addpoints]{exam}

% Header and footer.
\pagestyle{headandfoot}
\runningheadrule
\runningfootrule
\runningheader{CS 113 Discrete Mathematics}{Homework VII}{Late Spring 2018}
\runningfooter{}{Page \thepage\ of \numpages}{}
\firstpageheader{}{}{}

\boxedpoints
\printanswers
\usepackage[table]{xcolor}
\usepackage{amsfonts,graphicx,amsmath,hyperref,amssymb}
\hypersetup{
    colorlinks=true,
    linkcolor=blue,
    urlcolor=cyan,
}

\title{Habib University\\CS-113 Discrete Mathematics\\Late Spring 2018\\HW 7}
\author{$<your-id>$}  % replace with your ID, e.g. rk03734
\date{Due: Whenever you want}


\begin{document}
\maketitle

\begin{questions}



\question
In the DM Final Exam, some students were mistakenly allowed to went out of the exam hall so they can fulfill their obligations in the washrooms. Dr. Waqar Saleem has realized his mistake and now wants your help to find out, from all those who went out, which ones really cheated. Now, Waqar has given you a hint: the cheaters used their phones! 

Use neural networks and unsupervised learning with regression algorithm to find out students who cheated based on their facebook and whatsapp profiles along with the search history that you extracted after hacking their phones.
  \begin{solution}
    
  \end{solution}

\question 
Two very well known complexity classes are \textit{P} and \textit{NP}. They can very simply be defined that are easy to solve (\textit{P}) and problems for which its easy to verify if a given to verify if a given solution is correct(\textit{NP}).

It has not been proven if \textit{P} is a sub-class of \textit{NP}. The CS department at Habib believes that the relation goes the other way around and \textit{NP} is a sub-class of \textit{P}. They believe this so much they think that an assignment that requires 2 weeks to solve need 4 weeks to be verified. \textbf{Prove by induction} that their claim holds for all assignments.


By \textit{Sami Murtaza}


  \begin{solution}
	
  \end{solution}
  
\question 
Question 3

  \begin{solution}
  
  \end{solution}

\question
Question 4

  \begin{solution}
			
  \end{solution}

\question 
Question 5

	\begin{solution}
		% Write your solution here
	\end{solution}


\end{questions}


\end{document}