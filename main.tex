%CS-113 S18 HW-7
%Released: 27-April-2018
%Deadline: 11-May-2018 7.00 pm
%Author: Mudasir Hanif Shaikh.
%Contributors: Farooq Abdul Rehman, Sami Murtaza. 

\documentclass[addpoints]{exam}


% Header and footer.
\pagestyle{headandfoot}
\runningheadrule
\runningfootrule
\runningheader{CS 113 Discrete Mathematics}{Homework VII}{Late Spring 2018}
\runningfooter{}{Page \thepage\ of \numpages}{}
\firstpageheader{}{}{}

\boxedpoints
\printanswers
\usepackage[table]{xcolor}
\usepackage{graphicx}

\usepackage{amsfonts,graphicx,amsmath,hyperref,amssymb}


\title{Habib University\\CS-113 Discrete Mathematics\\Late Spring 2018\\HW 07}
\author{$<your-id>$}  
\date{Due: 00h, 01 May 2018}


\begin{document}
	\maketitle
	\begin{questions}
		
		
		
		\question
		Institutional researchers at the Thrones University had been conducting a research over the course of the last semester in which they recorded all the responses of Professors to their students' queries. After the first few weeks, they realized that there is a Professor, named Prof.W, whose responses are particularly interesting to them. They noted down all his responses and found that, to most queries, his response was either `Ok', `Good Luck' or `It's in the syllabus'. \\ \\
		They continued with the research and interestingly enough, found that the number of times Prof. W replied `Ok' to a query was much greater than the number of all the other responses combined. The number got so big that one researcher excitedly claimed that \\ \\ \textit{`The number of times Prof. W replied `Ok' to his students' queries during the course of the semester is uncountable.'} \\ \\
		Researches at the department of institutional research now look to you, Prof.W's discrete maths students, to prove (or disprove) the above claim \\
		(Hint: one possible way of going about this question can be, you guessed it right, PRIME NUMBERS!)
		\begin{solution}
			% Write your solution here
		\end{solution}
		
		
		\question  
		Two very well known complexity classes are \textit{P} and \textit{NP}. They can very simply be defined as problems that are easy to solve (are of the class \textit{P}) and problems for which its easy to verify if a given solution is correct(these belong to class \textit{NP}).\\
		It has not been proven if \textit{P} is a sub-class of \textit{NP}. The CS department at the Thrones University believes that the relation goes the other way around and \textit{NP} is a sub-class of \textit{P}. So much so, they think that an assignment that requires 2 weeks to solve needs 4 weeks to be verified. \textbf{Prove by induction} that their claim holds for all assignments. \\
		
		\begin{solution}
			
		\end{solution}
		
		
		\question
		In the Discrete Mathematics Final Exam at the Thrones University, some students were unnecessarily allowed to go out of the exam hall so that they could get done with their `important' business. However, this did not go as planned and Prof.W soon realized that something was wrong. But due to an exhausting semester, Prof.W is unwilling to do the hard work of catching the fishes, so he wants you, his students, to figure out who cheated on the final. He's given you a hint: Anyone who used their phone and/or went to the washroom is a suspect. \\ 
		
		Using the deductive skills that you acquired in previous assignments, find out the students who cheated based on their Facebook and WhatsApp profiles along with the search history that you got from the IT department. Support your findings by constructing a proof.
		\begin{solution}
			
		\end{solution}
		
		
		\question
		Discrete Mathematics Professors at the Thrones University claim that: \\ \\ 
		\textit{`Assignments are just an extension of the stuff covered in class and they only assess how well students can apply the concepts that they are taught in class'} \\ \\
		You, being a discrete maths student at the Thrones University, believe that this claim is invalid based on the `aWeSome' assignment questions that you were made to solve. Can you disprove your professors' claim using proof by counterexample? (You only have the previous assignments to use as an example). 
		\begin{solution}
			
		\end{solution}
		
		\question 
		Student X from Prof Y's section claims to be from the future. He thinks that this assignment, like all the previous assignments, makes no sense. Using any of the proof techniques you have learned in this course so far, prove that student X's claim is valid (or invalid)
		
		\begin{solution}
		\end{solution}
		
		
	\end{questions}
	
	
\end{document}
